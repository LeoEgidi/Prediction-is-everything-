\documentclass[12pt]{article}
\usepackage{amsmath}
\usepackage{graphicx}
\usepackage{enumerate}
\usepackage{natbib}
\usepackage{url} % not crucial - just used below for the URL 
\usepackage[table]{xcolor}
\usepackage[caption=false]{subfig}
\usepackage{enumerate}
\usepackage{comment}
\usepackage{float}

%\pdfminorversion=4
% NOTE: To produce blinded version, replace "0" with "1" below.
\newcommand{\blind}{0}
\usepackage[colorinlistoftodos,textwidth=2.2cm, textsize=tiny]{todonotes}
\setlength{\marginparwidth}{2.5cm}
\newcommand{\leo}[1]{\todo[linecolor=orange, size=\footnotesize, backgroundcolor=orange!25,bordercolor=orange]{Leo: #1}}
\newcommand{\jonah}[1]{\todo[linecolor=blue, size=\footnotesize, backgroundcolor=blue!25,bordercolor=orange]{Jonah: #1}}


% DON'T change margins - should be 1 inch all around.
\addtolength{\oddsidemargin}{-.5in}%
\addtolength{\evensidemargin}{-1in}%
\addtolength{\textwidth}{1in}%
\addtolength{\textheight}{1.7in}%
\addtolength{\topmargin}{-1in}%




\begin{document}

%\bibliographystyle{natbib}

\def\spacingset#1{\renewcommand{\baselinestretch}%
{#1}\small\normalsize} \spacingset{1}


%%%%%%%%%%%%%%%%%%%%%%%%%%%%%%%%%%%%%%%%%%%%%%%%%%%%%%%%%%%%%%%%%%%%%%%%%%%%%%

\if1\blind
{
  \title{\bf Title}
  \author{Leonardo Egidi\thanks{Dipartimento di Scienze Economiche, Aziendali, Matematiche e Statistiche `Bruno de Finetti',
	Universit\`{a} degli Studi di Trieste, Italy
    }}\hspace{.2cm}\\
    and \\
    Jonah Sol Gabry \thanks{Department of Statistics, Columbia University, New York, USA} \\
  \maketitle
} \fi

\if0\blind
{
  \bigskip
  \bigskip
  \bigskip
  \begin{center}
    {\LARGE\bf Prediction isn't everything, but everything is prediction}
\end{center}
  \medskip
} \fi

\bigskip
\begin{abstract}

Write an abstract  \\

\end{abstract}

\noindent%
{\it Keywords:}  prediction; explanation; OTHER KEYWORDS?????
\vfill

\newpage
\spacingset{1.9} % DON'T change the spacing!

\section{Introduction}
\label{sec:intro}


\section{Everything is prediction}
\label{sec:everything-prediction}

\subsection{Why is everything prediction?}

- Strong claim: everything (or more or less everything) in inferential statistics can be reframed through lens of prediction!  \\
- Explanation via prediction \\
- Goal of Bayesian modeling should NOT be to find posterior, but rather to find posterior predictive distribution \\
- Parameters don't exist, must connect to observables \\
- Even model checking/development can be viewed as prediction exercises (prior pred checks, posterior pred checks) \\

\subsection{Examples (especially including potential objections)}

- Hypothesis testing (Billheimer 2019 example) \\
- Ability estimation (e.g., IRT) \\
- Treatment effect estimation \\
- Causal inference \\
- What else?    


\section{Prediction isn't everything}
\label{sec:prediction-not-everything}

- Don't worry statisticians, we don't only care about prediction \\
- We still care about understanding how a model works, not just predictive accuracy \\
- Etc. 

\section{Discussion}
\label{sec:concl}

Discuss

\bigskip
\begin{center}
{\large\bf SUPPLEMENTARY MATERIAL}
\end{center}

%\begin{description}
%
%\item[Title:] Brief description. (file type)
%
%\item[R-package for  MYNEW routine:] R-package �MYNEW� containing code to perform the diagnostic methods described in the article. The package also contains all datasets used as examples in the article. (GNU zipped tar file)
%
%\item[HIV data set:] Data set used in the illustration of MYNEW method in Section~ 3.2. (.txt file)
%
%\end{description}

\section{Open points}

By reviewing the blog post and the comments:

- Don Rubin actually wrote something about the missing data-approach? Ask Andrew

- Paul Harrison. "prediction is something out of the knowledge of the statistician". Be careful with the definition.

- Christian Hennig comment: the attitude that ?everything is prediction? severely underestimates that we do science not only for prediction of what happens in specific situations, but also to inform our general thinking with implications on communication, creativity, problem solving, and decision making in situations that go beyond the more narrow prediction problems that are usually addressed by specific studies (or resulting from reframing such studies as treating prediction problems if they don?t do that explicitly). OUR RESPONSE: Yes, this is why "prediction isn't everything"! So we should make sure to emphasize this. 



\bibliographystyle{Chicago}
\bibliography{predbib}
\end{document}
