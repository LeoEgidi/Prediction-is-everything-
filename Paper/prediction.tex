\documentclass{statsoc}
\usepackage[english]{babel}
\usepackage[a4paper]{geometry}
\usepackage{bm}
\usepackage{amsmath}
\usepackage{amssymb}
%%\usepackage{graphics}
\usepackage[authoryear]{natbib}
\usepackage{relsize}
%\usepackage{color}
\usepackage[table]{xcolor}
\usepackage{multirow}
\usepackage{mathptmx}
%\usepackage{transparent}
%\usepackage{xcolor}
%\usepackage{efbox}
\usepackage{float}
%\usepackage{subfig}
%
%\usepackage{hvfloat}
%\usepackage{booktabs}
%\usepackage[font=footnotesize]{caption}
\usepackage{etoolbox}
\usepackage{url}
\usepackage[table]{xcolor}
\usepackage{array}
\usepackage{tikz}
\usetikzlibrary{trees}
\DeclareGraphicsExtensions{.pdf,.png,.jpg,.eps} 


\makeatletter
\patchcmd{\@makecaption}
{\parbox}
{\advance\@tempdima-\fontdimen2\font\parbox} % decrease the width!
{}{}
\makeatother  

\usepackage{graphicx}
\usepackage[caption=false]{subfig}
\usepackage{enumerate}
\usepackage{comment}
\usepackage{float}

\newcommand\Pair[4]{%
  \arrayrulecolor{cyan!60!black!40}%
  \arrayrulewidth=1pt
  \renewcommand\extrarowheight{1.5pt}%
  \begin{tabular}{|p{2cm}|>{\centering\arraybackslash}p{10pt}|}
  \hline
  \rowcolor{cyan!60!black!10}\textcolor{red!60!black}{#1} & \textcolor{red!60!black}{#2} \\
  \hline 
  \rowcolor{cyan!60!black!10}\textcolor{red!60!black}{#3} & \textcolor{red!60!black}{#4} \\
  \hline
  \end{tabular}%
}





\newtheorem{thm}{Theorem}

\title[]{Prediction is not everything, but everything is prediction}
\author[Egidi and Gabry]{Leonardo Egidi}
\address{Dipartimento di Scienze Economiche, Aziendali, Matematiche e Statistiche `Bruno de Finetti',
	Universit\`{a} degli Studi di Trieste,
	Trieste,
	Italy.}
\email{legidi@units.it}
\author[Egidi and Gabry]{Jonah Sol Gabry}
\address{Department of Statistics, Columbia University, New York,
USA.}
\email{jgabry@gmail.com}
 
 


\begin{document}

\maketitle

\begin{abstract}


\end{abstract}

\section{Introduction}

\section{Prediction for science or science for prediction?}

\subsection{It is prediction part of the science design?}

\color{black}

The main stages required to formulate a scientific law are summarized by \cite{russell2017scientific} as follows:
(1) observation of some relevant facts;
(2) formulation of a hypothesis underlying and explaining these facts;
(3) deduction of some consequences from this hypothesis.
Russell argues that to perfectly apply the scientific method, we should collect some facts $A, B, C,D$ and, by induction, formulate a general law, whose $A, B, C, D$ are examples. If this general law is true, we should then retrieve the same facts by deduction.  
In his opinion, the modern scientific method is born with Galileo Galilei, father of the law of falling bodies, and with Johannes Kepler, who discovered the three laws of planetary motion:
 \begin{quote}
Scientific method, as we understand it, comes into the
world full-fledged with Galileo (1564-1642), and, to a
somewhat lesser degree, in his contemporary, Kepler
(1571-1630). [...] They proceeded from observation of
particular facts to the establishment of exact quantitative
laws, by means of which future particular facts could be
predicted.
\end{quote}
%
As Russel states, Galilei provided a generalization of his theory by considering only a few observations, and further experiments confirmed the beauty and the correctness of his 
hypothesis. Kepler had formulated his theory by looking at the planets' motions. Isaac Newton (1642-1726) brought the theories of Galilei and Kepler together in his encompassing 
law of universal gravitation. When Albert Einstein (1879-1955) generalized the Newton's law in his theory of general relativity, the world was shocked by a sort of \emph{final 
theory} about the universe. Thus, in the last 500 years, physics---and, more generally, science---advanced by falsification and generalization of the previous theories, by providing new and more exciting theories to predict new facts.

However, the link of prediction to scientific laws is in our opinion more ambiguous than what people are usually inclined to think. Is prediction a central step in science? A naive 
answer could be: no, prediction is not explicitly part of the formulation of a scientific hypothesis (1)--(3), as drawn by Russell. Is prediction a relevant aim of science? A naive 
question could be: yes, scientists formulate quantitative laws, `by means of which future particular facts could be
predicted', and can then validate the goodness of their assumptions by somehow  measuring the predictive accuracy. The first answer could be in disagreement with some 
\emph{instrumentalist} scientists, who would claim that, from an instrumental perspective, predictive success is not merely \emph{symptomatic} of scientific success, but it is also 
\emph{constitutive} of scientific success \citep{hitchcock2004prediction}. A more sophisticated answer could be:  prediction is not explicitly part of the formulation of a 
scientific hypothesis (1)--(3) \emph{at the time the law is posed}, but it becomes relevant and relevant as the science advances. The chain of events which brought Newton to 
generalize the theories of Galilei and Kepler first, and Einstein to revisit the gravitational law of Newton then, was supposedly based on the fallacy of some predictions, and it gained sense 
only \emph{ex-post}. The fact that the bodies in proximity to the earth surface were revealed by Newton to not fall exactly with a constant acceleration---the acceleration slightly 
rises as they get closer to the earth---did not make the Galilei's law of constant acceleration for falling bodies less scientific, or scientifically totally wrong. Scientific falsification by mean of wrong predictions \citep{popper2005logic} is a powerful and exceptional tool, but we feel to warn the scientific community about its abuse/misuse. 

Over the last decades, scientific predictions have been popular not only in the context of physics and physical science, but for social sciences as well. Many algorithms and models 
have been developed to predict political scenarios, policies effects, sport results, fluctuations of national Gross Domestic Product (GDPs), and many others. The role played by 
prediction for social sciences is more obscure \citep{popper1944poverty, popper1945poverty} and much controversial, though data scientists are every day more and more asked to build 
`weapons of mass prediction' in many social contexts. The way in which they formulate their underlying theory about some data follows in the majority of the circumstances the scheme 
outlined by Bertrand Russell and reported at the beginning of this section: the stage of `hypothesis formulation' is vague here, but may be interpreted either in form of the 
classical statistical testing procedure, or in terms of a model to be checked. Though, the actual outcome may be far away from the predictions: Trump's win in the 2016 US 
Presidential Elections, Brexit, and Leicester's Premier League's win were very low-probability events, but they occurred. Can all of these rare events falsify the finest algorithms and models designed to not predict their occurrence? Our naive and tentative answer is no, they can't. On the other hand, a statistical procedure that had foreseen Leicester winning the Premier League at the beginning of the season 2015-2016 would have been a very bad model.

As statisticians and (data) scientists, demanded to build models for social and physical sciences, our efforts should be addressed to produce good algorithms/models, and make them falsifiable upon a strong check \citep{gelman2013philosophy}. Our skepticism regards the role of prediction in falsifying our models, for such a reason we would claim to be \emph{weak instrumentalists}: predictions and predictive accuracy are a central task of science, but only sometimes they are constitutive of scientific success.

%,: (out-of-sample) predictions are a task of science, but only (in-sample) predictions are constitutive of scientific success.

\subsection{Prediction as a confirmation theory approach}

%\textcolor{blue}{Popper, Kuhn, Mayo}

For Popper \citep{popper2005logic}, a theory is scientific only if is falsifiable, where the falsification of a theory is meant to be the the possibility to compare its predictions 
with the observed data. In his view, theories whose predictions conflict with any observed evidence must be rejected: prediction corroborates (or confirms) a theory when it survives 
an attempt at falsification; prediction delegitimizes a theory when it does not pass the falsification test.

The confirmation nature of prediction is crucial in natural sciences, such as physics. In general,  as \cite{hitchcock2004prediction} argue,   mathematical descriptions of the 
invariant behaviour of a physical phenomenon---such as Newton's and Keplero's laws, or Maxwell's equations---are essentially predictive: further experiments and observations can validate these theories. 

A well-known historical example of predictive confirmation in chemistry dates back to the middle of the 19th century---see \cite{maher1988prediction} for a detailed version of the 
example. At that time, more than 60 chemical elements were known, and new ones continuing to be discovered. Some prominent chemists attempted to determine their atomic weights, 
density and other properties, by collecting many experimental observations. In 1871, the Russian chemist Dmitri Mendeleev noticed that arranging the elements by their atomic 
weights, valences and other chemical properties tended to show a periodical recurrence. He found some gaps in the pattern, and he argued that these missing values corresponded to 
some existing elements which had not yet been discovered: he named three of these elements eka-aluminium, eka-boron, and eka-silicon, by giving some detailed description of their 
properties. Despite the skepticism of the scientific community, the French Paul-Emile Lecoq de Boisbaudran in 1874, the Swedish Lars Fredrik Nilson in 1878, and the German Clemens 
Winkler in 1886 discovered three elements which corresponded to descriptions of eka-aluminium, eka-boron, and exa-silicon, respectively: these three elements are better known 
now as gallium, scandium and germanium. The predictive ability of Mendeleev was remarkable---the Royal Society awarded him the Davy Medal in 1882---, and the new discovered elements well represent pieces of evidence which confirmed the theory.

Predictive confirmation is still ambiguous in social sciences. As argued by \citep{popper1944poverty, popper1945poverty} and \cite{sarewitz1999prediction}, social sciences 
have long tried to emulate physical sciences in developing invariant mathematical laws of human behaviour and interaction to predict economics quantities, elections, policies, etc.; 
many scholars agreed about the fact that a social theory should be judged on its power to predict \citep{friedman1953essays}. 

However, we believe that social science predictions require more and more motivations to validate the underlying theory. In the 2016 United States Presidential Election the Republican Donald Trump defeated the Democrat Hillary Clinton by winning the Electoral College (304 vs 227), but gaining lower voters' percentage (46.1\% vs 48.2\%). According to various online poll aggregators, Hillary Clinton was given a 65\% or 80\% or 90\% chance of winning the electoral college.  As \cite{gelman2016elections} argues:

\begin{quote}
 These probabilities were high because Clinton had been leading in the polls for months; the probabilities were not 100\% because it was recognized that the final polls might be off by quite a bit from the actual election outcome. Small differences in how the polls were averaged corresponded to large apparent differences in win probabilities; hence we argued that the forecasts that were appearing, were not so different as they seemed based on those reported odds. The final summary is that the polls were off by about 2\% (or maybe 3\%, depending on which poll averaging you’re using), which, again, is a real error of moderate size that happened to be highly consequential given the distribution of the votes in the states this year.
\end{quote}
%
In November 2016, many modelers, included Nate Silver, the founder of the well-known FiveThirtyEight blog (\url{https://fivethirtyeight.com}), failed to predict the Trumps' win. However, it is naive to conclude that those models failed because their underlying mechanism was wrong; rather, the political science predictions cannot  entirely act as theory's confirmation tools, due to many reasons attributed, for instance, to nonresponse and voters' turnout, as explained by \cite{gelman2016elections2}:

\begin{quote}
Yes, the probability statements are not invalidated by the occurrence of a low-probability event. But we can learn from these low-probability outcomes. In the polling example, yes an error of 2\% is within what one might expect from nonsampling error in national poll aggregates, but the point is that nonsampling error has a reason: it’s not just random. In this case it seems to have arisen from a combination of differential nonresponse, unexpected changes in turnout, and some sloppy modeling choices. It makes sense to try to understand this, not to just say that random things happen and leave it at that.
\end{quote}

\section{The role of prediction in statistics}

Statistics has always been thought as the \emph{science of inference}, or \emph{science of estimates}, and inference is always seen as separate from prediction. Inference is based on 
an underlying mathematical model for the data-generating process \citep{bzdok2018points}, its main task is to describe an unknown mechanism working through generalization: the 
inferential laws should in fact be as broad as possible, ideally valid for the population of interest, and not symptomatic of the observed data (it is out of the scope of this paper to review the distinct inferential 
approaches). Prediction moves from the observed to the unobserved, being the action designed to forecast future events without requiring a full understanding of the data-
generation process. Each person is more or less confident with the weather's predictions or with presidential election predictions, but rarely that person is aware of the underlying 
statistical model required to produce that forecast, unless he is a statistician/data scientist. In such a view, inference seems hard and obscure, and prediction easy and close 
to the people. This is often a paradoxical argument, since the inference is often associated to the \emph{explanation} of the problem, and should be relevant and available to the majority of the population

As statisticians, we are often faced with a double task. First, we must create a sound mathematical model to accomodate the data and retrieve useful inferences for our 
parameters---there is not distinction here between classical and Bayesian statistics, they are both designed to draw inferential conclusions, either in form of point estimates/
confidence intervals or in terms of posterior quantiles/credibility intervals. Then, we should use this model to make predictions, but this is rarely accounted by the statisticians in a transparent way.

For illustration purposes only, we consider the classical linear regression model, where $y_n$ denotes the response variable, $X$ is the $n \times p$ predictor matrix, and $\alpha,\beta_1,\beta_2,\ldots,\beta_p$ are the $p+1$ parameters---the intercept $\alpha$ and the $p$ regression parameters, we have

\begin{equation}
y_n = \alpha + \sum_{k=1}^p \beta_j x_{nk} + \epsilon_n, \ n=1,2,\ldots,N,
\label{eq:linear}
\end{equation}
%
with $\epsilon_n \sim \mathcal{N}(0, \sigma^2_{\epsilon})$. Once the model has been estimated and we have retrieved some parameters' estimates $\hat{\alpha}, \hat{\beta_1}, \beta_{2},\ldots,\hat{\beta}_p$, we could use them to provide a point forecast $\tilde{y}_{n+1}$ for the unobserved ${y}_{n+1}$ associated to the predictor vector $\tilde{x}_{n+1k}$. To account for uncertainty, rather than using a point forecast, we could also compute the prediction interval for $\tilde{y}_{n+1}$.

Once the value $y_{n+1}$ is known, the statistician can validate his prediction and check its plausibility. This \emph{ex post} validation is not available when fitting the model, and as such should not represent the plausibility of the model fit itself, since the model construction did not require $y_{n+1}$. In this misalignment between the fitting of the model without the $n+1$-th unit and the forecast validation of $y_{n+1}$ there is space for an infinite debate about the scientific role of prediction.


\color{blue}

Qui si potrebbe parlare dei limiti della previsione statistica, della non capacità di rendersi competitivi

Parlare della pistola
\color{black}



\subsection{Overfitting and data accomodation}

Even a simple linear regression case poses many challenges: should the statistician use all the $N$ data to build a reasonable/useful model, or could he take only a portion of the sample to accomodate the model (the training set) and use the remaining values to validate the model (the test set)? This apparently naive question pushed many scholars to debate about the presumed supremacy of prediction over accomodation \citep{maher1988prediction, hitchcock2004prediction, worrall2014prediction}. According to this competition, the statistician should ask himself whether he wants models that are true---or  approximately true---or predictively accurate. 

It is well-known that more complex models tend to yield poor predictive results

\color{blue}

Qui il discorso va portato a termine, magari non portandolo troppo per le lunghe, visto che si tratta di concetti molto noti

\color{black}


\subsection{The falsificationist Bayesianism framework}

\cite{gelman2013philosophy} argue that a key part of Bayesian data analysis regards the model checking through posterior predictive checks. In such a view, the prior is seen as a testable part of the Bayesian model and is open to falsification: from such intuition, \cite{gelman2017beyond} name this framework \emph{falsificationist Bayesianism}.

As stated by \cite{gelman2013bayesian}, the process of Bayesian data analysis can be idealized by dividing it into the following three steps:

\begin{enumerate}
\item Setting up a full probability model—--a joint probability distribution---for all observable and unobservable quantities 
           in a problem. The model should be consistent with knowledge about the underlying scientific problem and the data collection 
           process.
\item Conditioning on observed data: calculating and interpreting the appropriate posterior distribution, i.e. the conditional probability 
           distribution of the unobserved quantities of ultimate interest, given the observed data.
\item Evaluating the fit of the model and the implications of the resulting posterior distribution: how well does the model fit the 
           data, are the substantive conclusions reasonable, and how sensitive are the results to the modeling assumptions in step 1? 
           In response, one can alter or expand the model and repeat the three steps.
\end{enumerate}
%
In the above paradigm, predictions are never mentioned. But this does not mean that predictions are not relevant in the Bayesian paradigm. Denoted by $\tilde{y}$ the unobserved vector of future values, we may derive the posterior predictive distribution as

\begin{equation}
p(\tilde{y}|y) = \int p(\tilde{y}|\theta)p(\theta|y) d\theta,
\label{eq:ppdist}
\end{equation}
%
where $p(\theta|y)$ is the posterior distribution for $\theta$, whereas $p(\tilde{y}|\theta)$ is the likelihood function for future observable values. In the linear regression case~\eqref{eq:linear}, the posterior predictive distribution for the future observation $\tilde{y}_{n+1}$ is given by:

$$ p(\tilde{y}_{n+1}|y)= \int \mathcal{N}(\alpha+\sum_{k=1}^{p}\beta_p \tilde{x}_{n+1k}, \sigma^2_{\epsilon}) p(\alpha, \beta_1, \beta_2,\ldots,\beta_p|y) d\alpha  d\beta_1 d\beta_2\ldots d\beta_p.$$
%
Equation~\eqref{eq:ppdist} may be resambled in the following way:

\begin{equation}
p(\tilde{y}|y) = \frac{p(\tilde{y},y)}{p(y)}= \frac{1}{p(y)}\int p(\tilde{y},y,\theta)d\theta.
\label{eq:ppdist2}
\end{equation}
%
From Equation~\eqref{eq:ppdist2} we immediately notice that whenever we are interesting in predictions, we need to consider a joint model $p(\tilde{y},y,\theta)$ for both the observed data $y$ and the unobserved quantities $\tilde{y},\theta$. This joint model incorporates bot the likelihood and the prior, being $p(\tilde{y},y,\theta) = p(\tilde{y}|\theta)p(y|\theta)p(\theta)$. Thus, the joint model for the predictions, the data and the parameters is transparently posed, and open to falsification when the observable $\tilde{y}$ becomes known.

\subsection{Communication duties}

\color{blue}
Forse questa parte potrebbe andare anche nelle conclusioni e nel testo, e non per forza come sezione a se stante

\color{black}


\section{The role of prediction in machine learning}

%\color{blue}
%
%Breiman, Bzdok, Popper (we could argue that ML procedures are not falsifiable!)

\color{black}

As brilliantly argued by \cite{breiman2001statistical}, there are two cultures in the use of statistical modeling to reach conclusion from data: the data modeling culture, and the 
algorithmic modeling culture, also named machine learning (ML) culture. The two approaches differ in the underlying idea of how to explain the response variable $y$--stochastic data 
model vs. predictors function---and in the validation phase---goodness-of-fit tests vs. predictive accuracy. It is out of the scope of this paper to cover in more detail the two modeling classes, it is enough to be aware of the differences.

\color{blue}

Fare qualche esempio di classification tree, reandom forest

Parlare della scelta del training set e del test set, procedure spesso arbitrarie e non falsificabili.

Parlare del fatto che le procedure ML puntano davvero a validarsi secondo la previsione futura (strong instrumentalism), mentre le procedure statistiche modellistiche no.

Il paradosso sta nel fatto che allora, se le procedure di ML sono falsificabili mediante nuove previsioni, vi deve essere, sottostante all'analisi, una procedura da falsificare. Si, ma quale?? I metodi ML sono sempre data-driven, la loro falsificazione è generata dalla natura dei dati. Quindi, nascerebbero come procedure fortemente strumentaliste, ma in realtà non sono falsificabili.

Parlare del bazooka

\color{black}

\section{Going beyond inference and prediction: a tentative unifying approach}

\color{blue}

Forse questa sezione si può mettere dopo la attuale 3.2, in quanto ne sarebbe naturale appendice.

\color{black}

\section{Applied examples}

\color{blue}
Mondiali 2018. Scelte di dataset ben diverse portano a risultati ben diversi. Pensare alla relazione con la falsificazione.

\color{black}

\section{Conclusions}





\bibliographystyle{chicago}
\bibliography{predbib}

\end{document}